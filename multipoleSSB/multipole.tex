\documentclass[12pt]{article}
\usepackage[utf8]{inputenc}
\usepackage[margin=1in]{geometry} 
\usepackage{mathtools}
\usepackage{xcolor}

\newcommand{\note}[1]{\textcolor{red}{#1}}
\newcommand{\ex}[1]{\left\langle#1\right\rangle}
\newcommand{\nn}{\nonumber\\}
\renewcommand{\th}[1]{\frac{1}{#1}}
\newcommand{\half}{\th{2}}
\newcommand{\IR}{\text{IR}}
\renewcommand{\l}{\ell}
\newcommand{\cl}{\text{cl}}
\newcommand{\q}{\text{q}}
\newcommand{\goesto}{\rightarrow}

\title{CMW for Multipole Systems}
\author{charlie}
\date{May 2021}

\begin{document}

\maketitle

Throughout this document I will use $d$ for space dimensions and $D$ for spacetime dimensions.

\section{Review monopole systems}

Consider the Hamiltonian
\begin{align}
H &= \int d^dx\, \left( \partial_i\sigma \partial_i\sigma + m^2|\sigma|^2 + \lambda|\sigma|^4 + \cdots \right),
\end{align}
with the symmetry $\sigma(x)\goesto e^{i c} \sigma(x)$. There are two phases. For $m^2>0$ we have a disordered phase with $\ex{\sigma}=0$ and for $m^2<0$ we have an ordered phase with $\ex{\sigma} \ne 0$. 

The diagnostic for symmetry breaking is the correlation function,
\begin{align}
C(x) &= \ex{\sigma(x)\sigma(x)} - \ex{\sigma(x)\sigma(0)}\dots
\end{align}

Deep in the ordered phase, we can ignore longitudinal fluctuations and write $\sigma(x) = e^{i\phi(x)}$, with
\begin{align}
H &= \int d^d x(\partial_i \phi \partial_i \phi),
\end{align}
which now has an honest shift symmetry $\phi(x)\goesto \phi(x)+c$. This is the Hamiltonian we use to evaluate the validity of the ordered phase. However, it is only valid deep in the ordered phase!

\section{Effective field theories}

Euclidean field theory.

For monopole symmetries, the relationship between the original and ssb effective field theories is straightforward. Say we have a field $\sigma(x)\in \mathbf{C}$ and a symmetry $\sigma(x)\goesto e^{i c} \sigma(x) $, with $c\in \mathbf{R}$ a constant. \note{This is NOT a shift symmetry. Does it make sense to call it a monopole symmetry?} Say the Hamiltonian is
\begin{align}
H &= \int d^dx\, \left( \partial_i\sigma \partial_i\sigma + V(|\sigma|) \right)\nn
&= \int d^dx\, \left( \partial_i\sigma \partial_i\sigma + m^2|\sigma|^2 + \lambda|\sigma|^4 + \cdots \right),
\end{align}
so that when $m^2>0$ the symmetry is unbroken. Now let $m^2<0$ and let the potential be deep enough that $\sigma$ can effectively be described as $\sigma=e^{i\phi(x)}$.

The effective Hamiltonian for $\phi$ is 
\begin{align}
H &= \int d^dx\,\left( \partial_i\phi \partial_i\phi + \cdots \right),
\end{align}
which now has a monopole symmetry that \emph{is} a shift symmetry, $\phi(x)\goesto \phi(x)+c$

On the other hand, if we start with the Hamiltonian
\begin{align}
H &= \int d^dx \left( \partial_i\partial_j \sigma \partial_i \partial_j \sigma + \cdots \right),
\end{align}
then the Hamiltonian in terms of $\phi$ is
\begin{align}
H &= \int d^dx \big( (\partial_i \partial_j \phi) (\partial_i \partial_j \phi) + (\partial_i \phi \partial_j \phi) (\partial_i \partial_j \phi) + (\partial_i \partial_j \phi) (\partial_i \phi \partial_j \phi) + (\partial_i \phi \partial_j \phi) (\partial_i \phi \partial_j \phi) \big),
\end{align}
which has a bunch of interaction terms. \note{Does that matter?}

\note{Does Gromov care about symmetry breaking? How do Griffin et al. handle this?}

\section{Quantum systems}

Ref.~\cite{Griffin2015} considers nonrelativistic quantum systems with the action
\begin{align}
S = \int dt\, d^dx \left\{ \half \dot{\phi}^2 -\half \zeta_n^2 (\partial_{i_1} \dots \partial_{i_n} \phi)^2 + \cdots \right\},
\end{align}
where $\cdots$ represents higher order terms. Such an action has a dynamical critical exponent $z = n$. Although this exponent is fixed, we will leave factors of $z$ explicit for clarity.

Since the action is dimensionless and $[dt\,d^dx] = -z-d$ in units of spatial momentum, we have 
\begin{align}
[\phi] = \frac{d-z}{2},\qquad [\zeta_n] = z-n = 0,
\end{align}
so that $\phi$ is dimensionless at $d=z=n$. Ref.~\cite{Griffin2015} then shows that the $\phi$ field cannot exist at or below $d=n$ as follows.

Consider the propagator for the $\phi$ field,
\begin{align}
\ex{\phi(t,x) \phi(0)} = \int\frac{d\omega\, d^dk}{(2 \pi)^{d + 1}} \frac{e^{ik\cdot x -\omega t}}{\omega^2 + k^{2d} + \mu^{2d}},
\end{align}
where $\mu$ is an IR regulator. For large $|x|$ we can set $t=0$, so that
\begin{align}
\ex{\phi(t,x) \phi(0)} &= \int \frac{d^dk}{(2 \pi)^{d + 1}}e^{ik\cdot x} \int_{-\infty}^{\infty} \frac{d \omega}{\omega^2 + (k^d + \mu^d)^2}\nn
&= \pi \int \frac{d^dk}{(2 \pi)^{d + 1}} \frac{e^{ik\cdot x}}{k^{d}+\mu^{d}}\nn
&= \th{2(2\pi)^d}\int d^d\l\, \frac{e^{i\mu\,\l\cdot x}}{\l^d + 1}\nn
&= \frac{\Omega_d}{2(2\pi)^d}\int_0^\infty \l^{d-1} d\l \int \frac{d\phi}{2\pi}\frac{e^{i\mu|x|\l\cos \phi}}{\l^d+1}\nn
&= \th{(4\pi)^{d/2}\Gamma(d/2)}\int_0^\infty \frac{d \l\, J_0( \mu|x|\l)}{\l+ \l^{1-d}},
\end{align}
which \note{I still don't know how to do}. After some manipulations,
\begin{align}
\ex{\phi(t,x) \phi(0)} &= -\th{(4\pi)^{d/2} \Gamma(d/2)}\log(\mu|x|),
\end{align}
matching Eqn.~2.25 in Ref.~\cite{Griffin2015}.

Personally, I prefer to introduce $\mu$ as a cutoff, so
\begin{align}
\ex{\phi(t,x) \phi(0)} &= \int \frac{d^dk}{(2 \pi)^{d + 1}}e^{ik\cdot x} \int_{-\infty}^{\infty} \frac{d \omega}{\omega^2 + k^{2d}}\nn
&\sim \int \frac{d^dk}{(2 \pi)^{d + 1}} \frac{e^{ik\cdot x}}{k^d}\nn
&\sim \int_\mu^\Lambda \frac{dk}{2\pi} \frac{J_0(k |x|)}{k}\nn
&\sim -\log(\mu|x|),
\end{align}
where we have ignored constants.

Regardless, the point is that the lower critical dimension is $d=z=n$, and that it doesn't matter how we introduce the IR regulator.


\section{Nontrivial dynamical critical exponents}

Usually, we talk about a correspondence between \begin{itemize}
    \item phase transitions in classical systems with $d_\cl$ space dimensions at finite temperature
    \item phase transitions in quantum systems with $d_q=d_\cl-1$ space dimensions at zero temperature.
\end{itemize}
The first step in the argument is to Wick rotate the quantum theory to obtain a Euclidean theory in $d_q+1$ space dimensions. That theory can then be reinterpreted as a classical theory where all dimensions are spatial.

A key part of the argument was that the Euclidean theory was isotropic, because the quantum theory was relativistic. If the original quantum theory instead had a nontrivial critical exponent $z\in \textbf{Z}, z\ne 1$, the argument is different.

The most naive solution is to let $d_\cl = d_\q + z$. Let's see if that works.

\section{Classical systems}

Compare to Griffin.

The Hamiltonian we care about is
\begin{align}
\beta H &= \half\int d^dx\, (\partial_{ i_1} \dots \partial_{ i_n}\phi) (\partial_{ i_1} \dots \partial_{ i_n}\phi) \nn
&= -\half\int d^dx\, \phi(\partial^2)^n \phi,
\end{align}
and we will set $\beta=1$.

\subsection{Naive energy scaling}

Start with the clean case, with no disorder. Then there is no energy gain from forming domains. Say we form domains anyway, with size $L$. In the original $n=1$ case the order parameter varies linearly over a size $L$, say in the $z$ direction. The energy cost is
\begin{align}
\Delta H &= \int d^{d-1}x \int dz (\partial_z \phi)^2\nn
&= L^{d-1} \int dx \th{L^2}\nn
&= L^{d-2},
\end{align}
so domains proliferate if $d\le 2$.

For larger $n$ we want to minimize $(\partial_z^n \phi)^2$. Naively, we can do this by gluing together regions where $\phi$ scales as an order-$n$ polynomial (\note{Can we do better?}). For example, for $n=2$ we would have
\begin{align}
\phi = \begin{cases}
    2\left(\frac{z}{L}\right)^2, & 0<z<\frac{L}{2}\\
    -2\left(\frac{z-L}{L}\right)^2+1, & \frac{L}{2}<z< 1
\end{cases},
\end{align}
so that $(\partial_z^2 \phi)^2=L^{-4}$ is constant. Then, the energy cost is
\begin{align}
\Delta H &= \int d^{d-1}x \int dz (\partial_z^n \phi)^2\nn
&= L^{d-1} \int dx L^{-2n}\nn
&= L^{d-2n},
\end{align}
so the critical dimension is $d=2n$. Recall that in the quantum case we had $d_\q=z=n$, so our naive energy scaling argument matches the naive prediction $d_\cl = d_\q+z = 2n$. Good!

Now say there is some Gaussian quenched disorder and the order parameter forms domains of size $L$. Regardless of $n$, the energy gains from these domains is $L^{d/2}$.  The gain and cost balance out when $d/2 = d-2n$, so the critical dimension is now $d=4n$. That seems large, but it matches the $n=1$ case so it's at least not crazy.

\subsection{Diverging correlation function}

We want to use the correlation function to diagnose long-range order. If the correlation function remains finite at large distance, then the phase is ordered.
The correlation function that we care about is  
\begin{align}
C(x) &= \half \ex{(\phi(x)-\phi(0))^2} =  \langle\phi(0)\phi(0) - \phi(x)\phi(0)\rangle\nn
&= \int\frac{d^dk}{(2\pi)^2} \frac{1-e^{ik \cdot x}}{k^{2n}} \nn
&= \int dk\, k^{d-1-2n} \int_0^{2\pi} \frac{d\phi}{2\pi}(1 - e^{ikx \cos\phi} )\nn
&= \int dk\, k^{d-1-2n} (1 - J_0(kx)),
\end{align}
where $J_0(kx)$ is a Bessel function and we're ignoring constants. 
%We care about the asymptotic forms $J_0(kx) = 1$ for $kx<<1$ and $J_0(kx) = \sqrt{2/(\pi kx)}\cos(kx-\pi/4)$ for $kx>>1$. 
% The infrared regulator enters either in the form of a cutoff $\mu_\IR$ or by modifying the propagator to
% \begin{align}
% G(x) = \int\frac{d^dk}{(2\pi)^d}\frac{e^{ik\cdot x}}{k^{2n} + \mu_\IR^{2n}},
% \end{align}
% where we now see that $\mu_\IR$ has the interpretation of mass only when $n=1$.
We will introduce an IR cutoff into the integral, rather than as a regulator.

If we take $n=1$ we should recover the CMW theorem,
\begin{align}
C(x) &= \int_\mu^\Lambda dk\, k^{d-3} (1 - J_0(kx)) \nn
&= x^{2-d} \int_{\mu x}^{\Lambda x} dy\, y^{d-3} (1 - J_0(y)) \nn
&= \begin{cases}
\Lambda^{d-2}, & d>2,\\
\log(\Lambda x) - \log(\mu x), & d=2,\\
x^{2-d}, & d<2,
\end{cases}
\end{align}
matching Griffin 2.21. \note{Redo for regulator analysis?}

For general $n$ everything goes through the same way and we get
\begin{align}
C(x) &= \int_\mu^\Lambda dk\, k^{d-1-2n} (1 - J_0(kx)) \nn
&= \begin{cases}
\Lambda^{d-2}, & d>2n,\\
\log(\Lambda x) - \log(\mu x), & d=2n,\\
x^{2-d}, & d<2n,
\end{cases}
\end{align}
so that now the critical dimension is $2n$, matching the energy scaling. The result is that above the critical dimension, the correlation function asymptotes to a finite value, so the symmetry must be spontaneously broken. Below the critical dimension, the correlation function diverges and the symmetry is preserved.

Now, in the case of quenched disorder we will follow the conventions of Imry and Ma~\cite{ImryMa} where
\begin{align}
\Delta H = -\int d^dx\, h(x) \sigma(x),
\end{align}
with $\sigma(x) = e^{i\phi(x)}$ and $h(x)\in \mathbf{C}$. Then 
\begin{align}
G_\perp(k) &= \th{k^{2n}},
\end{align}
and
\begin{align}
\ex{m_\perp(x)m_\perp(0)} &= \int d^dk\, G_\perp^2 e^{ik \cdot x}\nn
&= \int dk\, k^{d-4n-1}J_0(kx)
\end{align}
which diverges for $d\le 4n$, reproducing the naive energy scaling.

\bibliographystyle{unsrt}
\bibliography{big}
\end{document}
