\documentclass[12pt]{article}
\usepackage[utf8]{inputenc}
\usepackage[margin=1in]{geometry} 
\usepackage{mathtools}
\usepackage{xcolor}

\newcommand{\note}[1]{\textcolor{red}{#1}}
\newcommand{\ex}[1]{\left\langle#1\right\rangle}
\newcommand{\nn}{\nonumber\\}
\renewcommand{\th}[1]{\frac{1}{#1}}
\newcommand{\half}{\th{2}}
\newcommand{\IR}{\text{IR}}
\renewcommand{\l}{\ell}
\newcommand{\cl}{\text{cl}}
\newcommand{\q}{\text{q}}
\newcommand{\goesto}{\rightarrow}

\title{Project Overview}
\author{charlie}
\date{May 2021}

\begin{document}

\maketitle

I will use this document to keep track of the goals of this project and to keep track of the documents in this overleaf folder.

\section{The main goal}

Various phases that were thought to be beyond the symmetry-breaking paradigm have recently been described using the spontaneous breaking of exotic symmetries, such as subsystem symmetries, higher symmetries, and multipole symmetries. For some of these symmetries, CMW-like theorems exist in the literature. None have Imry-Ma-type theorems. 

Some systems, like the X-cube model, have alternative descriptions in terms of multipole or higher-form symmetries. This type of system can provide a non-trivial check on whatever symmetry-breaking theorems we come up with.

\section{Files}

\begin{itemize}
\item \texttt{multipole.tex} has some CMW and Imry-Ma theorems.
    
\item \texttt{dipole1d.tex} has attempts at finding a microscopic theory that supports partial multipole symmetry breaking.
\end{itemize}

\bibliographystyle{unsrt}
\bibliography{big}
\end{document}
