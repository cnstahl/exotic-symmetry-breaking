\documentclass[12pt]{article}
\usepackage[utf8]{inputenc}
\usepackage[margin=1in]{geometry} 
\usepackage{mathtools}
\usepackage{xcolor}

\newcommand{\note}[1]{\textcolor{red}{#1}}
\newcommand{\ex}[1]{\left\langle#1\right\rangle}
\newcommand{\nn}{\nonumber\\}
\renewcommand{\th}[1]{\frac{1}{#1}}
\newcommand{\half}{\th{2}}
\newcommand{\IR}{\text{IR}}
\renewcommand{\l}{\ell}
\newcommand{\cl}{\text{cl}}
\newcommand{\q}{\text{q}}
\newcommand{\goesto}{\rightarrow}

\title{Dipole mediated symmetry breaking}
\author{charlie}
\date{October 2021}

\begin{document}

\maketitle

Consider a system with an exact dipole symmetry, with fields $\phi$ and $\phi_\mu$ that transform as monopoles and dipole under the symmetry, respectively. They should have kinetic terms that look like $(\partial_\mu \partial_\nu \phi)^2$ and $(\partial_\mu \phi_\nu)^2$.  We can also allow them to have a coupling $g(\phi_\mu - \partial_\mu \phi)^2$


\end{document}
