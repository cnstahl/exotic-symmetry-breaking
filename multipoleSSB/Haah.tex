\documentclass[12pt]{article}
\usepackage[margin=1in]{geometry} 
\usepackage{mathtools}
\usepackage{xcolor}

\newcommand{\note}[1]{\textcolor{red}{#1}}
\renewcommand{\cal}{\mathcal}
\newcommand{\nn}{\nonumber\\}

\title{Multipole group for Haah's U(1) code}
\author{charlie}
\date{August 2021}

\begin{document}

\maketitle

\section{Maximal quadrupole group}

In 3d, the maximal quadrupole group is generated by the set
\begin{align}
\left\{T_i, R_{ij}, \cal{P}_0, \cal{P}_1^{I_1}, \cal{P}_2^{I_2},
\right\}
\end{align}
where $T_i$ and $R_{ij}$, with $i,j=1,2,3$, are the translation and rotations. $\cal{P}_0$ is the constant shift, $\cal{P}_1^{I_1}$ are dipole shifts with $I_1=1,2,3$, and $\cal{P}_2^{I_2}$ are quadrupole shifts with $I_2=1,\dots,6$.

\section{Haah group}

Now, the group consists of
\begin{align}
\left\{T_i, R_{(1,1,1)}, \cal{P}_0, \cal{P}_1^1, \cal{P}_1^2, \cal{P}_2^1, \cal{P}_2^2
\right\},
\end{align}
where $R_{(1,1,1)}$ is a rotation in the plane perpendicular to that vector. The included polynomial shift symmetries are 
\begin{align}
\cal{P}_1^1 &= x_1-x_2, \nn
\cal{P}_1^2 &= x_1+ x_2 - 2x_3, \nn
\cal{P}_2^1 &= (x_1 - x_2) (x_1 + x_2 - 2x_3)\nn
\cal{P}_2^2 &= (2x_1 - x_2 - x_3) (x_2 - x_3),
\end{align}
which is, of course, a subgroup of the maximal quadrupole group.

This \note{almost} contains the 2d maximal dipole group as a subgroup\dots

If we define new variables $x = (x_1 - x_2)/\sqrt{2}$, $y = (x_1 + x_2 - 2x_3)/\sqrt{6}$, and $z = (x_1 + x_2 + x_3)/\sqrt{3}$, then we can write the Haah group as 
\begin{align}
\cal{M_H} &= \left\{T_i, R_{z}, \cal{P}_0, \cal{P}_1^1, \cal{P}_1^2, \cal{P}_2^1, \cal{P}_2^2
\right\},\nn
\cal{P}_1^1 &= \sqrt{2} x , \nn
\cal{P}_1^2 &= \sqrt{6} y, \nn
\cal{P}_2^1 &= \sqrt{12} xy\nn
\cal{P}_2^2 &= (3\sqrt{2} x + \sqrt{6} y) (\sqrt{6} y - \sqrt{2} x)\nn
&= 6y^2 +2\sqrt{12} xy - 6x^2,
\end{align}
so that $y^2-x^2$ is included in the group, but $y^2+x^2$ is not. 

% \bibliographystyle{unsrt}
% \bibliography{big}
\end{document}
