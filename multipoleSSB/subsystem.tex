\documentclass[12pt]{article}
\usepackage[utf8]{inputenc}
\usepackage[margin=1in]{geometry} 
\usepackage{mathtools}
\usepackage{xcolor}

\newcommand{\note}[1]{\textcolor{red}{#1}}
\newcommand{\ex}[1]{\left\langle#1\right\rangle}
\newcommand{\nn}{\nonumber\\}
\renewcommand{\th}[1]{\frac{1}{#1}}
\newcommand{\half}{\th{2}}
\newcommand{\IR}{\text{IR}}
\renewcommand{\l}{\ell}
\newcommand{\cl}{\text{cl}}
\newcommand{\q}{\text{q}}
\newcommand{\goesto}{\rightarrow}

\title{Subsystem vs Multipole}
\author{charlie}
\date{October 2021}

\begin{document}

\maketitle

For subsystem symmetries we have the conservation law
\begin{align}
\frac{d}{dt} \int_{y=a} dx \; \rho = \frac{d}{dt} \int_{x=b} dy \; \rho = 0.
\end{align}
This is equivalent to the conservation law
\begin{align}
0 &= \frac{d}{dt} \int dxdy \, \big ( f(x) + g(y) \big )\rho,
\end{align}
for arbitrary functions $f$ and $g$. We can then Taylor expand these functions. This makes the conservation law look like a conservation law for a multipole symmetry where the polynomials are $x, x^2, x^3, \dots$ and $y, y^2, y^3, \dots$


\end{document}
