\documentclass[12pt]{article}
\usepackage[margin=1in]{geometry} 
\usepackage{mathtools}
\usepackage{xcolor}

\newcommand{\note}[1]{\textcolor{red}{#1}}
\renewcommand{\cal}{\mathcal}
\newcommand{\nn}{\nonumber\\}

\title{Nontrivial dispersion relations}
\author{charlie}
\date{September 2021}

\begin{document}

\maketitle

\section{Symmetries and dispersions}

Monopole symmetry means the kinetic term cannot see constant shifts, so the invariant derivative has to be $D\phi = \partial_i \phi$. Then the kinetic term is $(\partial_i\phi)^2$, and the propagator is $k_ik_i = k^2$. In general, if there are no solutions to Eqns.~28-31 in Gromov's paper, then we have to use the maximal kinetic term $(\partial_{i_1}\dots \partial_{i_n})$, which is invariant under any transformation from $\mathcal{M}_\text{max}^{n-1}$. Then the kinetic term is $(\partial_{i_1}\dots \partial_{i_n} \phi)^2$, the propagator is $(k^2)^n$, and the (classical) critical dimension is $d=2n = 2(a+1)$, the result Gromov quotes from the Griffin paper. 

Gromov points out that Eqns.~28-31 are overdetemined and it is not clear when they have a solution. However, they do sometimes have a solution. For example, consider the dipole group extended by transformations of the form $\delta \phi = \mu_{ij}x^ix^j$ where $\mu$ is traceless. Then the most-relevant kinetic term is $(\partial^2 \phi)^2$, so that the propagator is $(k^2)^2$. Clearly, the classical critical dimension is $d=4$, in contrast to Gromov's claim that it should be $d=6$ since we include some polynomials of degree $2$. Gromov's claim is incorrect because he is citing the Griffin paper, where they always assume the maximal multipole group.

Can we make any more general statements about the correct critical dimension? We can divide the cases into isotropic and anisotropic as follows.

\section{CMW theorems}

\subsection{Isotropic}

The non-maximal, isotropic dispersion relations with smallest degree are the two parts of the dipole dispersion,
\begin{align}
E_1 &\sim (k^2)^2, \qquad \qquad E_2 \sim (k_i k_j)^2, \quad i\ne j.
\end{align}
Note that $(k^2)^2 = (k_ik_j)^2$, where the latter is the dispersion for the maximal dipole group. Furthermore, $(d-1)E_1=dE_2$. However, the two dispersions are built from different invariant derivatives using Gromov's prescription (Eqn. 36) and the requirement that the kinetic term be rotationally invariant.
Although these are dipole dispersion relations, they each individually admit a reduced but still isotropic quadrupole symmetry. Since they are constant multiples of $(k^2)^2$, they both share their critical dimension with the maximal dipole dispersion.

\subsection{Anisotropic}

Now, what about the dispersion relation $E = k^4 + p^2$, where $k$ has $d-1$ components and $p$ is a scalar? The correlation is
\begin{align}
C &\sim \int \frac{d^{(d-1)}k \, dp}{k^4 + p^2} \nn
&\sim \int \frac{d^{(d-1)}k}{k^2},
\end{align}
which diverges at $d=3$, which does not match the CMW theorem for any maximal dispersion relation. We can explain this result by assuming the field varies linearly in the $p$ direction and quadratically in the $k$ directions. The energy density will be $\sim L_p^{-2}$ or $L_k^{-4}$, depending on which direction it is varying in. The total energy cost is
\begin{align}
E &\sim L_p^{-2} L_k^{(d-1)} L_p + L_k^{-4} L_k^{(d-1)} L_p\nn
&\sim L_k^{(d-1)} L_p^{-1} + L_k^{(d-5)} L_p
\end{align}
which is minimized when $L_p \sim L_k^2$. Then $E\sim L_k^{d-3}$, so there is no energy cost at $d=3$, agreeing with the correlation calculation.


We can generalize this by letting $k$ have $q$ components and $p$ have $r$ components, with $d=q+r$. The correlation is
\begin{align}
C &\sim \int \frac{d^qk \, d^r\!p}{k^4 + p^2}\nn
&\sim \int d^qk \int \frac{p^{(r-1)}dp}{k^4+p^2} \nn
&\sim \begin{cases} \int d^qk \log k^4, &r=2\\
    \int \frac{d^qk}{k^2}, & r=1,
\end{cases}
\end{align}
so that if $r=2$ the symmetry can never be broken, and if $r=1$ then $q=2$ is critical. Of course, if $r=0$ we have an isotropic quartic dispersion and $q=4$ is critical. 

We can again appeal to the energy-scaling argument. This time, the total energy cost is
\begin{align}
E &\sim L_k^{q} L_p^{r-2} + L_k^{q-4} L_p^r,
\end{align}
so that $L_p \sim L_k^2$ as before, and $E\sim L_k^{q+2r-4}$. Combinations $(q,r)$ for which symmetry breaking cannot occur are
\begin{align}
&(0,0), (0,1), (0,2), \nn
&(1,0), (1,1), \nn
&(2,0), (2,1), \nn
&(3,0), \nn
&(4,0).
\end{align}
We see that the critical dimension $d=q+r$ can be 2, 3, or 4, depending on how many directions have quadratic or quartic dispersions.

\subsection{Haah}

In the Haah group, the three derivatives are
\begin{align}
D_1 &= \partial_1 + \partial_2 + \partial_3,\nn
D_2 &= \partial_1^2 + \partial_2^2 + \partial_3^2,\nn
D_3 &= \partial_1 \partial_2 + \partial_2 \partial_3 + \partial_3 \partial_1.
\end{align}
At this point, Gromov writes down the kinetic term as $g_{\alpha\beta} (D_\alpha \phi) (D_\beta \phi)$, and he also gauges it. Is there a unique way to turn this into a dispersion relation?

\end{document}